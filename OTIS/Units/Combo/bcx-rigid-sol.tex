\documentclass[11pt]{scrartcl}
\usepackage[usenames,dvipsnames,svgnames]{xcolor}
\usepackage[shortlabels]{enumitem}
\usepackage[framemethod=TikZ]{mdframed}
\usepackage{amsmath,amssymb,amsthm}
\usepackage{epigraph}
\usepackage[colorlinks]{hyperref}
\usepackage{microtype}
\usepackage{mathtools}
\usepackage[headsepline]{scrlayer-scrpage}
\usepackage{thmtools}
\usepackage{listings}
\usepackage{derivative}
\renewcommand{\epigraphsize}{\scriptsize}
\renewcommand{\epigraphwidth}{60ex}
\addtolength{\textheight}{3.14cm}
\ihead{\footnotesize\textbf{BCX-rigid}}
\ohead{\footnotesize Updated Wed 3 Jan 2024 02:39:32 UTC}
\providecommand{\clubs}[1]{$#1\clubsuit$}
\providecommand{\clubg}[1]{\bgroup\color{green!40!black}[$#1\clubsuit$]\egroup}

\providecommand{\ol}{\overline}
\providecommand{\eps}{\varepsilon}
\providecommand{\half}{\frac{1}{2}}
\providecommand{\dang}{\measuredangle} %% Directed angle
\providecommand{\CC}{\mathbb C}
\providecommand{\FF}{\mathbb F}
\providecommand{\NN}{\mathbb N}
\providecommand{\QQ}{\mathbb Q}
\providecommand{\RR}{\mathbb R}
\providecommand{\ZZ}{\mathbb Z}
\providecommand{\dg}{^\circ}
\providecommand{\ii}{\item}
\providecommand{\alert}{\textbf}
\providecommand{\opname}{\operatorname}
\providecommand{\ts}{\textsuperscript}
% hacks for arc
\providecommand{\tarc}{\mbox{\large$\frown$}}
\providecommand{\arc}[1]{\stackrel{\tarc}{#1}}
\reversemarginpar
\providecommand{\printpuid}[1]{\marginpar{\href{https://otis.evanchen.cc/arch/#1}{\ttfamily\footnotesize\color{green!40!black}#1}}}

\mdfdefinestyle{mdbluebox}{roundcorner=10pt,innerbottommargin=9pt,
    linecolor=blue,backgroundcolor=TealBlue!5,}
\declaretheoremstyle[headfont=\sffamily\bfseries\color{MidnightBlue},
    mdframed={style=mdbluebox},]{thmbluebox}
\mdfdefinestyle{mdredbox}{frametitlefont=\bfseries,innerbottommargin=8pt,
    nobreak=true,backgroundcolor=Salmon!5,linecolor=RawSienna,}
\declaretheoremstyle[headfont=\bfseries\color{RawSienna},
    mdframed={style=mdredbox},headpunct={\\[3pt]},postheadspace=0pt,]{thmredbox}
\mdfdefinestyle{mdgreenbox}{linecolor=ForestGreen,backgroundcolor=ForestGreen!5,
    linewidth=2pt,rightline=false,leftline=true,topline=false,bottomline=false,}
\declaretheoremstyle[headfont=\bfseries\sffamily\color{ForestGreen!70!black},
    mdframed={style=mdgreenbox},headpunct={ --- },]{thmgreenbox}
\mdfdefinestyle{mdblackbox}{linecolor=black,backgroundcolor=RedViolet!5!gray!5,
    linewidth=3pt,rightline=false,leftline=true,topline=false,bottomline=false,}
\declaretheoremstyle[mdframed={style=mdblackbox}]{thmblackbox}
\declaretheorem[style=thmredbox,name=Problem]{problem}
\declaretheorem[style=thmredbox,name=Required Problem,sibling=problem]{reqproblem}
\declaretheorem[style=thmbluebox,name=Theorem,numberwithin=problem]{theorem}
\declaretheorem[style=thmbluebox,name=Lemma,sibling=theorem]{lemma}
\declaretheorem[style=thmbluebox,name=Theorem,numbered=no]{theorem*}
\declaretheorem[style=thmbluebox,name=Lemma,numbered=no]{lemma*}
\declaretheorem[style=thmgreenbox,name=Claim,sibling=theorem]{claim}
\declaretheorem[style=thmgreenbox,name=Claim,numbered=no]{claim*}
\declaretheorem[style=thmblackbox,name=Remark,sibling=theorem]{remark}
\declaretheorem[style=thmblackbox,name=Remark,numbered=no]{remark*}
\declaretheorem[style=thmgreenbox,name=Definition,sibling=theorem]{definition}
\declaretheorem[style=thmgreenbox,name=Definition,numbered=no]{definition*}
\declaretheorem[style=thmblackbox,name=Example,sibling=theorem]{example}
\declaretheorem[style=thmblackbox,name=Example,numbered=no]{example*}

\newenvironment{walkthrough}{\noindent\textbf{\color{green!40!black}Walkthrough.}}{}
\newlist{walk}{enumerate}{3}
\setlist[walk]{label=\bfseries (\alph*)}

\usepackage{asymptote}
\begin{asydef}
size(8cm); // set a reasonable default
usepackage("amsmath");
usepackage("amssymb");
settings.tex="pdflatex";
settings.outformat="pdf";
import geometry;
void filldraw(picture pic = currentpicture, conic g, pen fillpen=defaultpen, pen drawpen=defaultpen) { filldraw(pic, (path) g, fillpen, drawpen); }
void fill(picture pic = currentpicture, conic g, pen p=defaultpen) { filldraw(pic, (path) g, p); }
pair foot(pair P, pair A, pair B) { return foot(triangle(A,B,P).VC); }
pair centroid(pair A, pair B, pair C) { return (A+B+C)/3; }
\end{asydef}

\newcommand{\goals}[2]{\bgroup
\sffamily\small \emph{Instructions}: Solve \clubg{#1}.
If you have time, solve \clubg{#2}.\egroup\par}

%% 426c616e6b204c615465587e
\begin{document}
\title{Submission for BCX-RIGID}
\subtitle{OTIS (internal use)}
\author{YOUR NAME HERE}
\date{\today}
\maketitle

\begin{example*}[\href{https://aops.com/community/p6580534}{TSTST 2016/4}, $0\clubsuit$]
  Prove that if $n$ and $k$ are positive integers
  satisfying $\varphi^k(n) = 1$, then $n \le 3^k$.
  (Here $\varphi^k$ denotes $k$ applications of the Euler phi function.)
\end{example*} \printpuid{16TSTST4}

\begin{walkthrough}
Let $a$, $b$, $c$, \dots, denote positive integers.
\begin{walk}
  \ii For positive integers $a$, $b$,
  show that $n = 2^a \cdot 3^b$ takes $a+b$ steps.
  \ii How many steps does each of $n = 2^a 5^b$,
  $n = 2^a 17^b$, $2^a 3^b 7^c$, $2^a 11^b$ take?
  \ii Show that $2^a 2017^b$ takes $a+9b$ steps.
  \ii Define the function $w \colon \NN \to \ZZ_{\ge 0}$
  by $w(ab) = w(a) + w(b)$, $w(2)=1$,
  and $w(p) = w(p-1)$ for odd primes $p$.
  Figure out the connection between the values of $w(p)$
  and the answer to your answer in (b).
  \ii By looking at $\nu_2$ prove the conjecture in (d).
  \ii Show that $w(n)$ is the number of steps required for $n$,
  if $n$ is even. What if $n$ is odd?
  \ii Show that $w(n) \ge \log_3 n$ by induction on $n \ge 1$.
  (The case where $n$ is composite is immediate,
  so the only work is when $n$ is prime.)
  \ii In fact, prove that the stronger
  estimate $n \le 2 \cdot 3^{k-1}$ holds (and is best possible).
\end{walk}
As a rigid problem, this is a chief example:
the point of the problem is to determine the function $w$,
and the ``extraction'' of comparing to $\log_3$ occurs at the end.
It's important to realize that $w$ is ``God-given'';
we were not permitted any decisions in deriving it.

It might be tempting to try and prove $\varphi(n) \ge n/3$
or similar statements, but this is false,
and in any case not representative of small cases.
However, I think trying the ``small cases'':
which in this situation are those $n$ with relatively
few prime factors --- suggests that this is the wrong approach.
\end{walkthrough}
%% You're not expected to write up walkthroughs (unless you really want to).
%% The source is just for your reference.

\begin{example*}[\href{https://aops.com/community/p9654050}{China TST 2018/2/3}, $0\clubsuit$]
  Two positive integers $p$ and $q$ are fixed.
  There is a blackboard with $n$ positive integers written on it.
  An operation is to choose two of the same number $a$, $a$
  written on the blackboard, and replace them with $a+p$, $a+q$.
  Determine the smallest $n$ (in terms of $p$ and $q$) for which
  such an operation could go on infinitely.
\end{example*} \printpuid{18CHNTST23}

\begin{walkthrough}
\begin{walk}
  \ii Reduce the problem to the case where $\gcd(p,q) = 1$.
  (Hint: if $d \coloneqq \gcd(p,q) > 1$, then work modulo $d$.)
  \ii Come up with a guess for the answer.
  It might be helpful to try $(p,q) = (n,1)$ and $(p,q) = (3,2)$.
  \ii It might seem right now the problem is not that rigid,
  because for example one gets to make a choice at each step of the process
  what numbers are supposed to be erased.
  But consider actually the smallest number on the board.
  What can you say about it?
  \ii Use your answer to (c) to eliminate the illusion of choice:
  prove that we can WLOG assume that we always operate on the smallest number.
  (You should assume $n$ is minimal for this part,
  meaning every entry is changed infinitely often.)
  \ii Assume that the smallest number on the board initially is $0$.
  Make sense of the following table,
  which is an example for $(p,q) = (3,5)$.
  \[
  \begin{bmatrix}
    0 & 0 \\
    3 & 5 & 3 \\
    6 & 5 & 8 & 5 \\
    6 & 10 & 8 & 8 & 6 \\
    9 & 10 & 8 & 8 & 11 \\
    9 & 10 & 13 & 11 & 11 & 9 \\
    12 & 10 & 13 & 14 & 16 & 14 \\
    12 & 10 & 13 & 14 & 16 & 14 & 10 \\
    12 & 15 & 13 & 14 & 16 & 14 & 13 & 12 \\ \hline
    15 & 15 & 13 & 14 & 16 & 14 & 13 & 17 \\
    15 & 15 & 18 & 14 & 16 & 14 & 16 & 17 \\
    15 & 15 & 18 & 17 & 16 & 19 & 16 & 17 \\
    18 & 20 & 18 & 17 & 16 & 19 & 16 & 17 \\
  \end{bmatrix}
  \]
  \ii Come up with a construction with the minimal number of elements
  for which the process goes on indefinitely.
  (For example, use engineer's induction on the last rows
  of the example above.)
  \ii Study the first nonzero numbers in each column
  of the above table, and come up with a guess for what they are
  in terms of $p$ and $q$ in general.
  We'll denote the set of such numbers by $S$.
  \ii Show that we set up the table above in such a way
  that the first two columns contain every element of $S$.
  \ii Prove that for every $s \in S$,
  there exists a column $C$ (not including the first two)
  such that $\max(S \cap C) = s$.
  \ii Use this to show that the minimum you conjectured is correct.
\end{walk}
This example is interesting because of part (c);
it's a surprising example of a problem which does not
look rigid at all when we first glance at it,
but then turns out to lose most of its degrees of freedom
after a bit of thinking.
In that way we get a problem which really only has two
parameters $p$ and $q$ and can make much progress
by studying the resulting table.
\end{walkthrough}
%% You're not expected to write up walkthroughs (unless you really want to).
%% The source is just for your reference.

\begin{example*}[\href{https://aops.com/community/p2745872}{TSTST 2012/8}, $0\clubsuit$]
  Let $n$ be a positive integer.
  Consider a triangular array of nonnegative integers as follows:
  \begin{center}
  \begin{asy}
  size(8cm);
  defaultpen(fontsize(10pt));
  label(scale(0.8)*"Row $1$:", (-6, 2.9));
  label(scale(0.8)*"Row $2$:", (-6, 2.2));
  label(scale(0.8)*"$\vdots$", (-6, 1.5));
  label(scale(0.8)*"Row $n-1$:", (-6, 0.8));
  label(scale(0.8)*"Row $n$:", (-6, 0.1));

  label("$a_{0,1}$", (0,2.8));

  label("$a_{0,2}$", (-1,2.1));
  label("$a_{1,2}$", ( 1,2.1));

  label(rotate(90)*"$\ddots$", (-2,1.5));
  label("$\vdots$", ( 0,1.5));
  label("$\ddots$", ( 2,1.5));

  label("$a_{0,n-1}$", (-3,0.7));
  label("$a_{1,n-1}$", (-1,0.7));
  label("$\dots$", (1,0.7));
  label("$a_{n-2,n-1}$", (3,0.7));

  label("$a_{0,n}$", (-4,0));
  label("$a_{1,n}$", (-2,0));
  label("$a_{2,n}$", (-0,0));
  label("$\dots$", (2,0));
  label("$a_{n-1,n}$", (4,0));
  \end{asy}
  \end{center}
  Call such a triangular array \textit{stable} if for every $0 \le i < j < k \le n$ we have
  \[ a_{i,j} + a_{j,k} \le a_{i,k} \le a_{i,j} + a_{j,k} + 1. \]
  For $s_1$, \dots, $s_n$ any nondecreasing sequence of nonnegative integers,
  prove that there exists a unique stable triangular array such that
  the sum of all of the entries in row $k$ is equal to $s_k$.
\end{example*} \printpuid{12TSTST8}

\begin{walkthrough}
We'll use induction on $n$, so essentially fixing the first $n-1$ rows
with the values of $s_1$, \dots, $s_{n-1}$.
\begin{walk}
  \ii Left-aligning the array, consider the example:
  \[
  \begin{bmatrix}
    2 \\
    4 & 1 \\
    5 & 3 & 1 \\
    5 & 3 & 1 & 0
  \end{bmatrix}
  \]
  Verify that this is a stable array with
  $(s_1, s_2, s_3, s_4) = (2,5,9,9)$.
  \ii Write down the unique stable arrays
  for $(s_1, s_2, s_3, s_4) = (2,5,9,x)$
  when $x \in \{10, 11, 12, 13, 14\}$.
  You should find that one entry changes at each step.
  The final array should be:
  \[
  \begin{bmatrix}
    2 \\
    4 & 1 \\
    5 & 3 & 1 \\
    7 & 4 & 2 & 1
  \end{bmatrix}
  \]
  \ii In part (b) you likely found that
  each of the entries in the bottom row interacts
  with another entry,
  and there was always exactly one entry that could be incremented by $1$.
  Find a natural way to construct a \emph{tournament}
  on the bottom row such that an entry can be incremented
  if and only if it is a source (all edges point away from it).
  \ii What property does this tournament have
  that ensures exactly one vertex is a source at each step?
  Verify your conjecture directly from the definition of a stable array.
  \ii Using what you have found,
  prove existence using induction on $s_n$.
  \ii Prove uniqueness of $s_n$.
  One way to do this is to again use induction,
  using again the tournament property you found in (d) a couple times.
  There is also a direct proof that is less motivated, by bounding.
\end{walk}
This problem is especially amenable to ``rigid''-type thinking due
to the guarantee that there exists a \emph{unique} stable array for each $(s_n)_n$.
Thus you are essentially guaranteed that all the examples you
come up with when playing with the problem are ``the'' examples.
This allows you to verify \emph{from data} for example
that the stable arrays going from $s_n \to s_n + 1$
are indeed just increasing one entry by one.
Part (b) of this walkthrough is critical,
because that is the data on which the later claims are based.
\end{walkthrough}
%% You're not expected to write up walkthroughs (unless you really want to).
%% The source is just for your reference.

\newpage

% ========================================
\section*{Practice problems}
\goals{32}{40}

\epigraph{The only thing I'm really good at is\dots\
That's it! BARKING! There really isn't much else in life!}
{Missile in \emph{Ghost Trick}}

\begin{problem}[\href{https://aops.com/community/p1219486}{Shortlist 1995}, $3\clubsuit$]
  For an integer $x \geq 1$,
  let $p(x)$ be the least prime that does not divide $x$,
  and define $q(x)$ to be the product of all primes less than $p(x)$.
  In particular, $p(1) = 2$.
  For $x$ having $p(x) = 2$, define $q(x) = 1$.
  Consider the sequence $x_0, x_1, x_2, \dots$
  defined by $x_0 = 1$ and
  \[ x_{n+1} = \frac{x_n p(x_n)}{q(x_n)} \] for $n \geq 0$.
  Find all $n$ such that $x_n = 1995$.
\end{problem} \printpuid{95SLS3}

\begin{proof}
THe solutions are all $n$,$\forall n \in \mathbb R$
\end{proof}

\begin{problem}[\href{https://aops.com/community/p8633268}{IMO 2017/1}, $3\clubsuit$]
  For each integer $a_0 > 1$, define the sequence $a_0$, $a_1$, $a_2$,
  \dots, by
  \[
  a_{n+1} =
  \begin{cases}
  \sqrt{a_n} & \text{if $\sqrt{a_n}$ is an integer,} \\
  a_n + 3 & \text{otherwise}
  \end{cases}
  \]
  for each $n \ge 0$.
  Determine all values of $a_0$ for which there is a number $A$
  such that $a_n = A$ for infinitely many values of $n$.
\end{problem} \printpuid{17IMO1}

%% Type your solution to IMO 2017/1 (\href{https://otis.evanchen.cc/arch/17IMO1/}{17IMO1}), proposed by Stephan Wagner (SAF) here ...

%% --------------------------------------------------

\begin{problem}[NICE 2021/1, $2\clubsuit$]
  The \textit{fibboican} sequence $a_1$, $a_2$, $\dots$
  is defined by $a_1 = a_2 = 1$, and for integers $k \geq 3$,
  \begin{itemize}
  \item $a_k = a_{k-1} + a_{k-2}$ if $k$ is odd
  \item $\frac {1}{a_k} = \frac {1}{a_{k-1}} + \frac {1}{a_{k-2}}$
  if $k$ is even.
  \end{itemize}
  Prove that, for each integer $m\ge 1$,
  the numerator of $a_m$ (when written in simplest form)
  is a power of $2$.
\end{problem} \printpuid{21NICE1}

%% Type your solution to NICE 2021/1 (\href{https://otis.evanchen.cc/arch/21NICE1/}{21NICE1}) here ...

%% --------------------------------------------------

\begin{problem}[MOP 2011, $2\clubsuit$]
  Let $S$ be a finite set with $|S| > 1$ and
  let $\mathcal P(S)$ denote the set of all subsets of $S$.
  A function $f \colon \mathcal P(S) \to \mathcal P(S)$
  is called \emph{united} if $f(\emptyset) = \emptyset$
  and $f(A) \cup f(B) = f(A \cup B)$ for any $A, B \in \mathcal P(S)$.
  Show that the number of united functions
  is a perfect power.
\end{problem} \printpuid{11MOPR32}

%% Type your solution to MOP 2011 (\href{https://otis.evanchen.cc/arch/11MOPR32/}{11MOPR32}) here ...

%% --------------------------------------------------

\begin{problem}[\href{https://aops.com/community/p22698309}{Shortlist 2020 C1}, $2\clubsuit$]
  Let $n$ be a positive integer.
  Find the number of permutations $a_1,\ \dots,\ a_n$
  of $1,\ \dots,\ n$ satisfying
  \[ a_1 \le 2a_2 \le \dotsb \le na_n. \]
\end{problem} \printpuid{20SLC1}

%% Type your solution to Shortlist 2020 C1 (\href{https://otis.evanchen.cc/arch/20SLC1/}{20SLC1}) here ...

%% --------------------------------------------------

\begin{problem}[\href{https://aops.com/community/p26491041}{Brazil EGMO TST 2023/3, added by Isken Kenzhebaev}, $2\clubsuit$]
  Let $n \ge 1$ be a fixed positive integer.
  Max and Lewis play a turn-based game with a pile of $n$ stones.
  Max makes the first move and must remove exactly $1$ stone.
  In each subsequent turn, the player to move may remove between $1$ and $t+1$
  stones, where $t$ is the number of stones removed by the previous player.
  A player wins if they remove the last stone.
  For each $n$, determine which player has the winning strategy.
\end{problem} \printpuid{23BRAEST3}

%% Type your solution to Brazil EGMO TST 2023/3, added by Isken Kenzhebaev (\href{https://otis.evanchen.cc/arch/23BRAEST3/}{23BRAEST3}) here ...

%% --------------------------------------------------

\begin{problem}[\href{https://aops.com/community/p741371}{Shortlist 2006 A1}, $3\clubsuit$]
  A sequence of real numbers $a_0$, $a_1$, \dots\
  is defined recursively by
  \[ a_{i+1} = \left\lfloor a_i \right\rfloor
  \cdot \left\{ a_i \right\} \]
  for $i \ge 0$, where $\{x\} = x - \left\lfloor x \right\rfloor$
  is the fractional part.
  Prove that $a_{i+2} = a_i$ for sufficiently large indices $i$.
\end{problem} \printpuid{06SLA1}

%% Type your solution to Shortlist 2006 A1 (\href{https://otis.evanchen.cc/arch/06SLA1/}{06SLA1}) here ...

%% --------------------------------------------------

\begin{reqproblem}[\href{https://aops.com/community/p12752847}{IMO 2019/5}, $5\clubsuit$]
  Let $n$ be a positive integer.
  Harry has $n$ coins lined up on his desk, which can show either heads or tails.
  He does the following operation: if there are $k$ coins which show heads and $k > 0$,
  then he flips the $k$th coin over; otherwise he stops the process.
  (For example, the process starting with $THT$ would be
  $THT \to HHT \to HTT \to TTT$, which takes three steps.)

  Prove the process will always terminate, and determine the average number of steps
  this takes over all $2^n$ configurations.
\end{reqproblem} \printpuid{19IMO5}

%% Type your solution to IMO 2019/5 (\href{https://otis.evanchen.cc/arch/19IMO5/}{19IMO5}), proposed by David Altizio (USA) here ...

%% --------------------------------------------------

\begin{problem}[\href{https://aops.com/community/p1187174}{Shortlist 2007 C1}, $5\clubsuit$]
  Let $n \ge 1$ be an integer. Find all sequences
  $a_1$, $a_2$, \dots, $a_{n^2 + n}$ consisting of $0$ and $1$ such that
  \[ a_{i+1} + a_{i+2} + \dots + a_{i+n}
  < a_{i+n+1} + a_{i+n+2} + \dots + a_{i+2n} \]
  for all $0 \le i \le n^2-n$.
\end{problem} \printpuid{07SLC1}

%% Type your solution to Shortlist 2007 C1 (\href{https://otis.evanchen.cc/arch/07SLC1/}{07SLC1}) here ...

%% --------------------------------------------------

\begin{problem}[\href{https://aops.com/community/p17829263}{Shortlist 2019 C1}, $5\clubsuit$]
  The infinite sequence $a_0$, $a_1$, \dots of integers
  satisfies the relations $0 \le a_n \le n$ and
  \[ \binom{n}{a_0} + \binom{n}{a_1} + \dots + \binom{n}{a_n} = 2^n \]
  for every nonnegative integer $n$.
  Show that every nonnegative integer appears in the sequence.
\end{problem} \printpuid{19SLC1}

%% Type your solution to Shortlist 2019 C1 (\href{https://otis.evanchen.cc/arch/19SLC1/}{19SLC1}) here ...

%% --------------------------------------------------

\begin{problem}[\href{https://aops.com/community/p3043748}{JMO 2013/4}, $3\clubsuit$]
  Let $f(n)$ be the number of ways to write $n$ as a sum of powers of $2$,
  where we keep track of the order of the summation.
  For example, $f(4)=6$ because $4$ can be written
  as $4$, $2+2$, $2+1+1$, $1+2+1$, $1+1+2$, and $1+1+1+1$.
  Find the smallest $n$ greater than $2013$ for which $f(n)$ is odd.
\end{problem} \printpuid{13JMO4}

%% Type your solution to JMO 2013/4 (\href{https://otis.evanchen.cc/arch/13JMO4/}{13JMO4}), proposed by Kiran Kedlaya here ...

%% --------------------------------------------------

\begin{problem}[\href{https://aops.com/community/p124444}{Shortlist 1998 N8}, $3\clubsuit$]
  Let $a_{0}$, $a_{1}$, $a_{2}$, \dots\ be an increasing sequence of nonnegative integers
  such that every nonnegative integer can be expressed
  uniquely in the form $a_{i}+2a_{j}+4a_{k}$,
  where $i$, $j$ and $k$ are not necessarily distinct.
  Determine $a_{1998}$.
\end{problem} \printpuid{98SLN8}

%% Type your solution to Shortlist 1998 N8 (\href{https://otis.evanchen.cc/arch/98SLN8/}{98SLN8}) here ...

%% --------------------------------------------------

\begin{problem}[\href{https://aops.com/community/p281572}{IMO 2005/2}, $3\clubsuit$]
  Let $a_1$, $a_2$, \dots\ be a sequence of integers
  with infinitely many positive and negative terms.
  Suppose that for every positive integer $n$
  the numbers $a_1$, $a_2$, \dots, $a_n$
  leave $n$ different remainders upon division by $n$.
  Prove that every integer occurs exactly once in the sequence.
\end{problem} \printpuid{05IMO2}

%% Type your solution to IMO 2005/2 (\href{https://otis.evanchen.cc/arch/05IMO2/}{05IMO2}), proposed by Nicholas de Bruijn (NLD) here ...

%% --------------------------------------------------

\begin{problem}[\href{https://aops.com/community/p10561203}{APMO 2018/4}, $5\clubsuit$]
  Let $ABC$ be an equilateral triangle.
  From the vertex of $A$ we draw a ray towards the interior of the triangle
  such that the ray reaches one of the sides of the triangle.
  When the ray reaches a side, it then bounces off following the \emph{law of reflection}.
  After $n$ bounces the ray returns to $A$ without ever
  landing on any of the other two vertices.
  Find all possible values of $n$.
\end{problem} \printpuid{18APMO4}

%% Type your solution to APMO 2018/4 (\href{https://otis.evanchen.cc/arch/18APMO4/}{18APMO4}) here ...

%% --------------------------------------------------

\begin{reqproblem}[\href{https://aops.com/community/p4725316}{EGMO 2015/2}, $5\clubsuit$]
  A \emph{domino} is a $2 \times 1$ or $1 \times 2$ tile.
  Determine in how many ways exactly $n^2$ dominoes can be placed
  without overlapping on a $2n \times 2n$ chessboard
  so that every $2 \times 2$ square contains at least
  two uncovered unit squares which lie in the same row or column.
\end{reqproblem} \printpuid{15EGMO2}

%% Type your solution to EGMO 2015/2 (\href{https://otis.evanchen.cc/arch/15EGMO2/}{15EGMO2}) here ...

%% --------------------------------------------------

\begin{reqproblem}[\href{https://aops.com/community/p4769949}{USAMO 2015/3}, $9\clubsuit$]
  Let $S = \left\{ 1,2,\dots,n \right\}$, where $n \ge 1$.
  Each of the $2^n$ subsets of $S$ is to be colored red or blue.
  (The subset itself is assigned a color and not its individual elements.)
  For any set $T \subseteq S$,
  we then write $f(T)$ for the number of subsets of $T$ that are blue.

  Determine the number of colorings that satisfy the following condition:
  for any subsets $T_1$ and $T_2$ of $S$,
  \[ f(T_1)f(T_2) = f(T_1 \cup T_2)f(T_1 \cap T_2). \]
\end{reqproblem} \printpuid{15AMO3}

%% Type your solution to USAMO 2015/3 (\href{https://otis.evanchen.cc/arch/15AMO3/}{15AMO3}), proposed by Gabriel Carroll here ...

%% --------------------------------------------------

\begin{problem}[\href{https://aops.com/community/p4663881}{APMO 2015/3, added by Haozhe Yang}, $5\clubsuit$]
  Find the smallest positive integer $n$ such that
  there exists a sequence $a_0$, $a_1$, \dots of real numbers such that:
  \begin{enumerate}[(i)]
  \ii $a_0$ is a positive integer;
  \ii for each nonnegative integer $i$ we have
  either $a_{i+1} = 2a_i + 1$ or $a_{i+1} =\frac{a_i}{a_i + 2}$.
  \ii $a_n = 2014$.
  \end{enumerate}
\end{problem} \printpuid{15APMO3}

%% Type your solution to APMO 2015/3, added by Haozhe Yang (\href{https://otis.evanchen.cc/arch/15APMO3/}{15APMO3}), proposed by Wang Wei Hua (HKG) here ...

%% --------------------------------------------------

\end{document}
